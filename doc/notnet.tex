\subsection{Not.Net language details}

Not.Net is a low level .NET language. It can be either used standalone or
embedded into MBase code.

\subsubsection{Types}

 Some short type names are defined for convenience: void, int, ptr, 
short, long, char, byte,
float, double, string, object, bool. A special type name 'this' must be used
as a reference to the current class (and it works even with an 
expression--embedded code).

 Other types must be named explicitly, as in C\#. Since 
normal MBase {\tt(dotnet ...)} function is used for a type lookup, its current
lookup path is taken into account, i.e. one can use 
{\tt(using (<namespaces>) ...)} construction.

\subsubsection{Statements}

 Not.Net is a statement--based language, so there is a distinction between 
statements and expressions.

 The following statements are defined:

{\tt (begin ...)} executes statements sequentially.

{\tt (quote <symbol>)} defines a label.

{\tt (for ((<symbol:name> <expr:initial> <expr:step>) <expr:condition>) ...)}
 is a simple looping statement: a variable {\tt <name>} is set as {\tt <initial>},
and until {\tt <condition>} is false, the body statements are evaluated and
{\tt <name>} is updated to {\tt <step>} value.

{\tt (while <expr:condition> ...)}
  loops until {\tt <condition>} is false, condition is checked prior to 
the body execution.

{\tt (dowhile <expr:condition> ...)}
  loops until {\tt <condition>} is false, condition is checked after the body
execution.

{\tt (foreach (<symbol:name> <expr:initial>) ...)}
  iterates over any {\tt System.Collection.IEnumerable} collection.

{\tt (goto <symbol:label>)}
  jumps to a given label.

{\tt (goto-if <expr:condition> <symbol:label>)}
 jumps to a given label if a condition value is true.

{\tt (goto-if-not <expr:condition> <symbol:label>)}
 jumps to a given label if a condition value is false.

{\tt (return <expr:value>)} returns a value from the current method.

{\tt (return)} returns from a void method.

{\tt (if <expr:condition> <statement:iftrue> [<statement:iffalse>])}
  executes {\tt iftrue} statement if condition value is true, and {\tt iffalse} otherwise. 

{\tt (try <statement:code> (catch (<type:exception> <symbol:name>) 
      <statement:excode>))} tries to execute {\tt <code>}, and if an {\tt <exception>} is raised, binds it to {\tt <name>} and executes {\tt <excode>}.

{\tt (throw <expr:value>)} throws and exception.

{\tt (<symbol:name> = <expr:value>)} defines a variable with a given 
  initial value. Variable type is same as {\tt <value>} type.

{\tt (<type> <symbol:name> = <expr:value>)} defines an explicitly typed 
  variable with a given initial value.

{\tt (<lvalue> <- <expr:value>)} destructively assigns a value to a given lvalue (e.g., a local variable, a field, an array element).

{\tt (lift-field <field>)} adds a field to the current class, this works
  for embedded not.net code as well as for complete class definitions.

{\tt (lift-method <method>)} adds a not.net method to the current class,
  method is defined as in {\tt not.class} language.

\subsubsection{Expressions}

 The following expressions are allowed:

{\tt ((<type>) <expr>)} casts an expression value to a given type

{\tt (type <type>)} loads a type token

{\tt (marshal <type> <expr>)} marshals an expression to a given type

{\tt (arr . <*expr>)} builds an array of given elements, array type if defined
  by the most generic type of all the expressions.

{\tt (arrt <type> . <*expr>)} builds an array of given elements, array type is
  specified explicitly.

{\tt (mkarr <type> <expr:length>)} build an array of a given type and
   dynamically evaluated length ({\tt <length>} must be an integer expression).

{\tt (ref <symbol>)} 

{\tt (aref <expr:array> <expr:index>)} references to an element of an array, 
 can be either an expression or an lvalue.

{\tt (begin <statement> ... <expr>)} executes a sequence of statements with a final expression.

{\tt (new <type> . <*expr>)} creates a new object or a value of a given type, using an appropriate constructor call.

{\tt (<type> \# <symbol:field>)} references to a static field.

{\tt (<expr> \# <symbol:field>)} references to a field.

{\tt (<expr> @ <symbol:method> . <*expr:args>)} calls a method.

{\tt (typeof <expr>)} gives a type of a value in runtime.

{\tt (istype <expr> <type>)} checks a type of a value

{\tt (\&\& <type> @ <symbol:method> . <*type:argtypes>)}

{\tt (\&\&\& <type> <type> <symbol>)}

{\tt (\&\&\& <type> <expr> <symbol>)}

{\tt null}

{\tt true}

{\tt false}

{\tt self}

{\tt <literal>}

{\tt (<binop> <expr> <expr>)}

{\tt (<unop> <expr>)}


\subsubsection{Class definition}

 Class is defined as follows:

{\tt (not.class <name> [(extends <type>)] [(implements <type>)...] ...)}

Class definition body may contain fields, methods and constructors definitions.

Field definition format is: {\tt (field <type> <name>  <attribute> ...)}, where attributes can be {\tt (public)}, {\tt (static)}, {\tt (private)}, 
      {\tt (protected)}.

Method definition is: 
{\tt (method (<attribute> ...) <type> <name> ((<argtype> <argname>) ...) ...)}

Constructor definition is:
{\tt (constructor (<attribute> ...) ((<argtype> <argname>) ...) ...)}



 
