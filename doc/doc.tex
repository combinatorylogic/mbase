\documentclass{article}
\usepackage{alltt}
\usepackage{listings}
\usepackage{rotating}
\lstloadlanguages{Lisp}
\lstset{language=Lisp,escapechar=�,commentstyle=\itshape}
\def\NET{\hbox{\tt .NET}}
\def\Lzero{${\cal L}_0$}
\def\Lone{${\cal L}_1$}
\def\LonePrim{${\cal L}_1'$}
\def\LoneC{${\cal L}_1^{\cal C}$}
\def\LoneCPrim{${{\cal L}_1^{\cal C}}'$}
\def\Csharp{\hbox{C\#}}

\newcommand{\mbfunction}[2]{\noindent {\bf Function:}~{\sl #1}~{#2} \par}

\newcommand{\mbmacro}[2]{{\noindent \bf Macro:}~{\sl #1}~{#2} \par}
\newcommand{\mbdefine}[1]{\noindent {\bf Definition:}~{\sl #1} \par}

\newcommand{\mbsection}[1]{\subsection{#1}}

\newcommand{\fixik}[1]{#1}

\title{MBase Reference Manual v1.0.2.0}
\author{Meta Alternative Ltd.}
\date{2014}


\begin{document}

\maketitle

\newpage

\tableofcontents

\newpage

\section{Introduction}

MBase is a framework for implementing compilers, code analysis tools and interpreters.
It consists of a simple ``general purpose'' core language and a hierarchy of domain specific languages
built on top of the core. This document describes the core language design and a part of the runtime library.


\section{Architecture outline}

MBase core contains several layers. Each layer forms a complete and
usable language, and every next layer is built as an extension to a
previous layer functionality. The most basic layer, runtime library
and \Lzero{} interpreter, is implemented in \Csharp. This layer is only used
for bootstrapping.

\subsection{Runtime library}

This is a minimalistic library written in \Csharp.  It contains a DLL
loading interface, a standard bootstrap sequence definition, and a
basic runtime functions set, including a minimal comprehensive set of
bindings to the {\tt System.Reflection} functionality. Some functions
of the runtime could have been omitted and implemented within a higher
level code, but they are here for the better \Lzero{} interpretation
efficiency. Functions defined via reflection are later redefined as
direct calls in a compiled mode (in \LonePrim).

\subsection{\Lzero{} interpreter}

It is a closure--based interpreter for the simple \Lzero{} language
and an S--expressions parser used by several bottom layers of MBase.
\Lzero{} statements are compiled to runnable objects, stacking into
one single object for one statement. E.g., {\tt (+ 2 2)} will be
represented as two objects returning a constant {\tt 2} and an object
applying the function {\tt +} to an array of evaluation results of two
constant objects.

Expression types are: Try, If, Apply, Sequence, Reference, Constant,
Lambda, Closure.

\subsection{\Lone $\to$ \Lzero{} compiler}

This is an \Lzero{} program, compiling \Lone{} expressions into
\Lzero.  It is compiled from \Lone{} itself, and used for
bootstrapping the whole system from scratch. Compiler contains a
simple macro expansion preprocessor, which is used later to extend the
language. Macros are removed from the bootstrap version of the compiler.

\subsection{\LonePrim{} language extensions}

After \Lone{} language is bootstrapped, it is extended (using macro
metaprogramming) to the level that provides better usability,
including a simple interface to \NET{} reflection, all standard
Lisp--like constructions, some basic pattern matching and lists
construction, some basic input$/$output (via reflected \NET{}
libraries).

\subsection{\LoneC$\to$CLI compiler}

Now, using the functionality of \LonePrim{} language, a ``native'' CLI
compiler is implemented for a superset of \Lone{}: \LoneC.  It
differs from \Lone: recursion and local variables are defined via
special constructions, while in \Lone{} they are macros which expands into
more basic forms.

\subsection{Alternative target languages}

At the level of \LoneCPrim, one can either use a direct \NET{} code
generation infrastructure or target the \LoneCPrim{} or its subsets
using macro metaprogramming or an interface to compiler
implementation. It is possible to mix both ways, since \LoneCPrim{}
have an embedded assebmly and \NET{} class generation macros.

The semantics of some languages requires different target language
semantics, often of a much higher level than the raw CLI or the
lisp--like \LoneCPrim. MBase provides some alternative target
semantics: it is a Forth--like stack language, Prolog (both
interpreted and WAM--compiled), a \Csharp--like \NET--specific
language, a lazy lambda evaluator (combinator graph reduction based)
and a finite state machines builder. Non--\NET{} targets are also
available, including an LLVM backend and an extensible C--like language.

\section{Metaprogramming support}

MBase macro system is somewhat similiar to the Common Lisp one, with
some advanced features added.

In Common Lisp--style macro expansion there's only one rule: if a head
of a list being processed by an expander is a symbol, and this symbol
is a name of a macro, a relevant macro function will be called and its
result will be processed by an expander again.  Very simple but
powerful approach, whereas one important thing is missing: any kind of
support for a context.  Macros are processed separately and normally
can't pass any information to each other.

MBase provides features to handle some sort of a context: lexically
scoped local macros and {\tt (inner-expand-first} {\tt ...)} and 
{\tt (inner-expand-with} {\tt ...)} special constructions. There is also
an in--list macro syntax.

Local macros are defined as follows:
\begin{verbatim}
(with-macros
  ((<macro-name> <macro-function>) ...)
  <body>)
\end{verbatim}

 Macro function is a function of one argument --- a list to be
 macro--expanded. Local macros are valid only in the scope of 
{\tt <body>}, and macro functions definitions are generated in the
 compilation time and interpreded by the macro expander (i.e., never
 compiled and would not appear in a binary module). If a local macro
 name shadows a global macro name or an outer scope local macro name,
 the latest (innermost) definition will be used.

 This feature can be used to deal with a context. To pass something
 stored in a local macro to an inner macro, that local macro must be
 expanded first. This is why we have introduced a special form 
{\tt inner-expand-first}. For example, an expansion of the following
 construction:

\begin{verbatim}
(inner-expand-first mymacro (list 1 2 3))
\end{verbatim}

 will go into an expansion of {\tt mymacro} with {\tt list} already
 expanded, i.e., a list {\tt (mymacro (cons 1 (cons 2 (cons 3 nil))))}
 will be actually expanded.

 This little trick allows to feed a macro with a content of a locally
 scoped macro. One of a practical examples of of this feature
 application is in the {\tt ast:visit} language defined in an extra
 libraries section below: local macros are defined for every node and
 every variant processing code, and macros like {\tt ast:mknode} use
 that information to substitute a correct format.

 {\tt with-macros} itself is a higher level feature not naturally
 known to a core macro expander. It is a macro built on top of a
 fundamental macro expansion control form: 
{\tt (inner-expand-with <hashtable> <code>)}. 
Hashtable is added to the list of a current
 context macros name tables and an expansion for an inner code is
 using this new environment, which is later discarded for an outer
 scope.

 There is no direct access to this environment from macro expanders,
 so the only way to fetch a data from it is {\tt inner-expand-first}.

\section{Library}

MBase library is staged as well as MBase itself. The same
functionality is provided by different layers, with different levels
of abstraction.

\subsection{Core library (accessible from \Lzero)}

Core functions are defined in \Csharp{} library code. Some of them are
later overridden by \LoneC{} native definitions. The following list is
not comprehensive, since we do not want to encourage MBase users to
rely on this lowest level.

\mbfunction{car}{(l)}

Returns a list's head.

\mbfunction{cdr}{(l)}

Returns a list's tail.

\mbfunction{cons}{($l_1${} $l_2$)}

Makes a new cons cell of head $l_1${} and tail $l_2$.

\mbfunction{null?}{(x)}

Checks if x is nil.

\mbfunction{list?}{(x)}

Checks if x is a list.

\mbfunction{string?}{(x)}

Checks if x is a string.

\mbfunction{symbol?}{(x)}

Checks if x is a symbol.

\mbfunction{char?}{(x)}

Checks if x is a char.

\mbfunction{number?}{(x)}

{Checks if x is a number.}

\mbfunction{boolean?}{(x)}

{Checks if x is a boolean.}

\mbfunction{eqv?}{(a b)}

{Checks if a physically equals to b.}

\mbfunction{eq?}{(a b)}

{Checks if a logically equals to b.}

\mbfunction{$>$}{(a b)}

{Checks if number a is greater than number b.}

\mbfunction{mkhash}{()}

{Makes an empty hashtable.}

\mbfunction{hashget}{(ht key)}

{Gets a hashtable value associated with a key.}

\mbfunction{hashput}{(ht key value)}

{Puts a value into a hashtable with a given key.}

\mbfunction{symbol-$>$string}{(value)}

{Converts a symbol into string}

\mbfunction{any-$>$string}{(value)}

{Converts any \NET{} object into string using {\tt ToString} method}

\mbfunction{string-$>$symbol}{(str)}

{Makes a symbol out of a given string. Symbol syntax is not enforced.}

\mbfunction{string-escape}{(str)}

{Enriches a string with proper escape characters}

\mbfunction{getfuncenv}{()}

{Returns a dictionary with all the current functions namespace definitions}

\mbfunction{getmacroenv}{()}

{Returns a hashtable with all the current macros namespace definitions}

\subsection{Boot library (\Lone{} language)}

This library is not covered by an automatic documenting system, so
here follows a brief description of the most important definitions.

\mbfunction{:Y1}{(f)}


 Y combinator for 1--ary functions.

 Usage example:

\begin{lstlisting}
(:Y1 (fun (factorial)
       (fun (n)
          (if (> n 1) (* n (factorial (- n 1))) 1))))
\end{lstlisting}

\mbfunction{:Y2}{(f)}

Y--combinator for 2--ary functions.

\mbfunction{:Y0}{(f)}

Y--combinator for 0--ary functions.

\mbmacro{:Yn}{(n)}

Generator for an $n$--ary Y--combinator.

\mbmacro{list}{args}

Makes a list of values


\mbfunction{append}{(a b)}

{Appends a list {\tt b} to the end of the list {\tt a}}

\mbfunction{to-string}{(value)}

{Converts a list or atom to string}

\mbfunction{read-int-eval}{(lst)}

{Compiles a list into \Lzero, evaluates it, returns a result}

\mbfunction{read-compile-eval-dump}{(lst)}

{Compile a list into \Lzero, evaluate it, return a string with a compiled code. Supposed to be used
for an initial bootstrapping purposes only.}

%%%%%%%%
{\raggedright
\input{clib}
}

\end{document}
